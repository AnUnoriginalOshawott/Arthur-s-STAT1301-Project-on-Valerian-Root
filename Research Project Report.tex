% ****** Start of file apssamp.tex ******
%
%   This file is part of the APS files in the REVTeX 4.1 distribution.
%   Version 4.1r of REVTeX, August 2010
%
%   Copyright (c) 2009, 2010 The American Physical Society.
%
%   See the REVTeX 4 README file for restrictions and more information.
%
% TeX'ing this file requires that you have AMS-LaTeX 2.0 installed
% as well as the rest of the prerequisites for REVTeX 4.1
%
% See the REVTeX 4 README file
% It also requires running BibTeX. The commands are as follows:
%
%  1)  latex apssamp.tex
%  2)  bibtex apssamp
%  3)  latex apssamp.tex
%  4)  latex apssamp.tex
%
\documentclass[10pt,preprintnumbers,amsmath,amssymb,floatfix,twocolumn,prl]{revtex4-2}
\usepackage[utf8]{inputenc}
\usepackage{amsmath}
\usepackage{mathrsfs}
\usepackage{graphics}
\usepackage{xspace}
\usepackage[T1]{fontenc}
\usepackage{graphicx}% Include figure files
\usepackage{dcolumn}% Align table columns on decimal point
\usepackage{bm}% bold math
%\usepackage{hyperref}
\usepackage{float}% add hypertext capabilities
%\usepackage[mathlines]{lineno}% Enable numbering of text and display math
%\linenumbers\relax % Commence numbering lines
\usepackage[capitalise]{cleveref}



\begin{document}


\title{A clinical test of the effects of Valerian root on waking hours at night}
\author{Arthur Souza Passos}
\email{a.passos@student.uq.edu.au}
\affiliation{School of Mathematics and Physics, The University of Queensland, QLD 4072, Australia}
\author{Charlie Boardman}
\email{c.boardman1@student.uq.edu.au}
\affiliation{School of Mathematics and Physics, The University of Queensland, QLD 4072, Australia}

\maketitle

\textit{Introduction}: Valerian root is a traditional plant primarily taken to promote sleep at night. Although a common traditional remedy, medical trials assessing the effectiveness of Valerian root have so far given mixed results. Hence, this study seeks to join this body of literature and help clarify whether Valerian root is an effective sedative. In particular, we will analyse the effects of continuous Valerian use over time on its effectiveness. \\

\textit{Methods}: 
We conducted a clinical trial of n=63 residents of the islands. We tried to take roughly equal numbers of male and female islanders from a variety of cities across the islands and from many age brackets (20-70 years of age) in order to distinguish the effects of the plant from age-related, sex-related, and geographical factors which may affect sleep. In line with conventional ethical standards, we consent was sought from all islanders we tested on. Since Valerian root is thought to be a sedative, we instructed all test subjects to reduce their consumption of alcohol during the trial (to prevent harmful interaction effects). Pregnant women were also excluded from our study to ensure their and their child’s safety. \

Once test subjects were found, each was randomly assigned either to a group of people given a 1g tablet of Valerian each night, or a control group who were given a sugar tablet. This was given 1-2 hours before the islanders went to sleep every night for 4 consecutive nights. To measure the pill’s effectiveness as a sedative, we examined the amount of hours each night that residents were awake from a hypnogram of each test subject’s sleep (to within ±10 mins due to scale on hypnograms) before beginning the trial, as well after 1-4 nights of the trial.  \\

If Valerian were an effective sedative, we would expect to see a statistically significant decrease in the waking hours of the islanders.





\begin{thebibliography}{99} 
% NOTE: This will get removed at some point. I prefer to use bibTeX for bibliography management (in particular, it can handle appropriately using APA formatting for us).
	\bibitem{Hotta} T. Koretsune and C. Hotta, Phys. Rev. B \textbf{89}, 045102 (2014).
	\bibitem{us} B. J. Powell, E. P. Kenny, J. Merino, Phys. Rev. Lett. \textbf{119}, 087204 (2017)
	\bibitem{Powell} B. J. Powell, \prb\xspace \textbf{96}, 174435 (2017).
	\bibitem{Haerter}  J. O. Haerter and B. S. Shastry,  Phys. Rev. Lett. \textbf{95}, 087202 (2005).
	\bibitem{Sposetti}  C. N. Sposetti, B. Bravo, A. E. Trumper, C. J. Gazza, and L. O. Manuel, Phys. Rev. Lett. \textbf{112}, 187204 (2014).
	\bibitem{RossReview} R. H. McKenzie,  Comments Cond. Mat. Phys. \textbf{18}, 309 (1998). 
	\bibitem{Kato} R. Kato, H. Kobayashi, H. Kim, A. Kobayashi, Y. Sasaki, T. Mori, and H. Inokuchi, Synth. Met. \textbf{B359} 27 (1988). 

 
\end{thebibliography}

\end{document}