% ****** Start of file apssamp.tex ******
%
%   This file is part of the APS files in the REVTeX 4.1 distribution.
%   Version 4.1r of REVTeX, August 2010
%
%   Copyright (c) 2009, 2010 The American Physical Society.
%
%   See the REVTeX 4 README file for restrictions and more information.
%
% TeX'ing this file requires that you have AMS-LaTeX 2.0 installed
% as well as the rest of the prerequisites for REVTeX 4.1
%
% See the REVTeX 4 README file
% It also requires running BibTeX. The commands are as follows:
%
%  1)  latex apssamp.tex
%  2)  bibtex apssamp
%  3)  latex apssamp.tex
%  4)  latex apssamp.tex
%
\documentclass[10pt,preprintnumbers,amsmath,amssymb,floatfix,twocolumn,prl]{revtex4-2}
\usepackage[utf8]{inputenc}
\usepackage{amsmath}
\usepackage{mathrsfs}
\usepackage{graphicx}
\usepackage{xspace}
\usepackage[T1]{fontenc}
\usepackage{graphicx}% Include figure files
\usepackage{dcolumn}% Align table columns on decimal point
\usepackage{bm}% bold math
\usepackage{hyperref}
\usepackage{float}% add hypertext capabilities
%\usepackage[mathlines]{lineno}% Enable numbering of text and display math
%\linenumbers\relax % Commence numbering lines
\usepackage[capitalise]{cleveref}

\begin{document}
\title{A clinical test of the effects of Valerian root on waking hours at night}
\author{Arthur Souza Passos}
\email{a.passos@student.uq.edu.au}
\affiliation{School of Mathematics and Physics, The University of Queensland, QLD 4072, Australia}
\author{Charlie Boardman}
\email{c.boardman1@student.uq.edu.au}
\affiliation{School of Mathematics and Physics, The University of Queensland, QLD 4072, Australia}

\maketitle
%%% WORDCOUNT: After excluding things which do not count towards the word count (e.g. citations), and condensing equations to one word (MS word seems to turn one equation into like 20 or so words), I'd say my word count is somewhere around 620 words%%%

\textit{Introduction}: Valerian root is a traditional plant primarily used medically to relieve insomnia and promote sleep \cite{ValerianSource1}. Although a common traditional remedy, medical trials assessing the effectiveness of Valerian root in promoting sleep are limited and have given mixed results \cite{ValerianSource2}. Hence, this study seeks to join this body of literature and help clarify whether Valerian root is an effective at this task. \\

\textit{Methods}: 
We conducted an experimental trial of $n=62$ residents on \emph{the Islands}. We tried to take roughly equal numbers of male and female islanders from a variety of cities across the islands and from many age brackets (20-70 years of age) to distinguish Valerian's effects from age-related, sex-related, and geographical factors which may affect sleep. Each test subject was randomly assigned to take either a 1g tablet of Valerian each night, or be given placebo sugar tablets each night (each with 50\% probability). This gave sample sizes $n_C = 30$ for the control group and $n_V = 32$ Valerian takers. Pills were given between 0.5-2 hours before the islanders' 10:00PM bedtimes (this ensures absoprtion of Valerian into the body \cite{ValerianSource1}) every night for 4 consecutive nights. \\

To measure Valerian's effectiveness in promoting sleep, we analysed if nightly hours of sleep meaningfully decreases in Valerian takers compared to the control group. Waking hours each night were measured from a hypnogram of each test subject’s sleep (within $\pm10$ mins due to graph's limited precision), both before beginning the trial, as well after 1-4 nights afterward. This limited number of nights prevents our result from being affected by the development of tolerances to Valerian's effects. \\

%%% NOTE: C and V are used as indices in the below working, not as powers %%%
Our test subjects' sleep habits over the trial may be influenced by their Valerian consumption, and external factors such as their age, personal health, and environment. Assuming linearity in the number of nights $t$ since trials began, the mean waking hours $y^C$ for the control group and $y^V$ for the Valerian takers will be:
$$y^C = \beta_0^C + \beta_1^C t, \qquad y^V = \beta_0^V + \beta_1^V t$$
For some parameters $\beta_0^C, \beta_0^V$ denoting baseline average waking hours before the trial began, and $\beta_1^C, \beta_1^V$ representing the rate per day of waking hour increases. If each test subject's waking time is independent of the others', and our data is normally distributed around our two lines with a variance independent of $t$, we can estimate these parameters and their standard deviations $\sigma_0^C, \sigma_1^C, \sigma_0^V, \sigma_1^V$ from a linear regression model. \\

If both the control and Valerian taker groups are representative of the islands' population (and hence the external factors they are under), the nightly changes $\beta_1^C, \beta_1^V$ should have no difference besides those due to Valerian's effects. Hence, we test the hypotheses:
$$H_0: \beta_1^C = \beta_1^V, \qquad H_1: \beta_1^C > \beta_1^V$$
Where the null hypothesis $H_0$ corresponds to Valerian having no effect on waking hours at night, and the alternative hypothesis $H_1$ corresponds to Valerian producing a statistically significant reduction in waking hours each night relative to the control group. Our estimates of $\beta_1^C, \beta_1^V$ should be $T$-distrubuted under our assumptions, with degrees of freedom $n_C - 2, n_V - 2$ respectively. Hence, assuming our null hypothesis, we expect the below to be $t_\text{df}$-distributed:
$$T = \frac{\beta_1^C - \beta_1^V}{\sqrt{\frac{(\sigma_1^C)^2}{n_C} + \frac{(\sigma_1^V)^2}{n_V}}}$$
Where the degrees of freedom (df) is given by Welch's approximation:
$$\text{df} = \frac{\left(\frac{(\sigma_1^C)^2}{n_C} + \frac{(\sigma_1^V)^2}{n_V}\right)^2}{\frac{(\sigma_1^C)^4}{n_C^2 (n_C - 2)} + \frac{(\sigma_1^V)^4}{n_V^2 (n_V - 2)}}$$
We then compute the probability of our data being more extreme than predicted from our $T$ value assuming $H_0$ as $p = \mathbb{P}(t_{df} \geq T)$ and evaluate our hypotheses against a standard 95\% confidence criterion. \\

\textit{Ethics}: Consent was obtained from all islanders we tested on. Since Valerian root may be a sedative (and hence interact with alcohol \cite{ValerianSource1}), all test subjects were instructed to reduce alcohol consumption during the trial. As little is known of valerian's effects on pregnancies \cite{ValerianSource2}, pregnant women were excluded from our study. \\

\textit{Results}: the results of our

\textit{Notes for discussion:} 
Here are some things which are worth noting (I'm just not sure if this is best to put in method or conclusion). This is NOT a section of the report, and should be edited out and incorporated into the report properly later
\begin{itemize}
\item Just looking at waking hours gives an incomplete characterisation of sedative-hypnotic drugs, as this does not analyse the quality of the persons' sleep (for example, there is evidence hypnotic-sedative drugs increase the prevalence of the N1 and N2 phases of deep sleep (its 2 shallower phases) while lowering the time spent in REM sleep and N3 deep sleep \cite{SleepCyclesSource}, disrupting the typical sleep cycle and giving lower quality sleep).

\item My preliminary data analysis is actually suggesting that Valerian is a stimulant, with Valerian takers experiencing a lower rate of decrease in waking hours compared to the placebo takers. This is fine -- we'll just need to p-test for Valerian being a stimulant (i.e. for null hypothesis $\beta_1^\text{Valerian} = \beta_1^\text{Control}$ and alternative hypothesis $\beta_1^\text{Valerian} > \beta_1^\text{Control}$). There may even be some historical precedent for Valerian being used as a stimulant in the 19th century \cite{ValerianSource2}, so this would be an interesting thing to talk about.

\item One person dropped out of my study, but he did so before I could do any proper measurements on him (I gave islanders about 24 hours to reduce their alcohol consumption before beginning trials). This may be worth mentioning in the report, but since no data ended being collected from him, I do not think it is the end of the world.

\item Using hypnograms instead of a survey helped prevent errors in the study from the islanders poorly estimating and/or lying about the hours of sleep they got, but it introduced the issue of me having to read the hypnograms in order to assign a sleeping time to them, which could not be done with perfect accuracy. I'd say sleep times are only really know to $\pm 10$ minutes, and is only given to a fidelity of 5 minutes.

\item I did not specifically select for islanders who were getting low amounts of sleep each night (I TRIED to with the islands' newspaper feature, but it would not work for any questions I wanted to ask about sleep schedules or anxiety/stress). This may make the effects of Valerian less pronounced here (if someone is already getting a healthy amount of sleep each night, the drug may not necessarily increase those hours). This could be worth discussing.
\end{itemize}

\bibliographystyle{apa}
\bibliography{ResearchProjectReportBibliography}

\end{document}
